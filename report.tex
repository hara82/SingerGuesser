\documentclass[a4paper,11pt]{article}


% 数式
\usepackage{amsmath,amsfonts}
\usepackage{bm}
\usepackage{here}
% 画像
\usepackage[dvipdfmx]{graphicx}
% hyperlink
\usepackage{hyperref}


\begin{document}

\title{Text-as-Data Learning Coursework}
\author{2822260H Ryosuke Hara}
\date{\today}
\maketitle

\section{Q1- Dataset}
\subsection{a) Overview of data}% 数式
The dataset used in this experiment is the lyrics of songs of famous singers.
Singers are chosen from a ranking in a website, \href{https://www.thefamouspeople.com/21st-century-singers.php}{"The greatest 21st century singers"}, to minimise the arbitrary choice of the author.

Lyrics were obtained by using lyricsgenius, a python API of Genius.com.
Since lyrics were fetched online, it includes some HTML tokens and some meta data about translation inside the lyrics.
The meta data about translation are supposed to be removed by preprocessing.


The main objective is to infer the singer from lyrics of a song.
I chose this dataset because I wanted to examine whether lyrics reflect singers' characteristics like the musical elements does.

Automatic classification have several applications.
It can be used for helping singers maintain the same trend among an album,
suggesting new singers for music listeners in recommendation system of music service like Spotify, 
detecting improper mimicking of the lyrics among singers, 

\subsection{b) Overview of input texts and labels}
summary of the labels and input text to be used \\

There are ten labels and each label is a name of a singer.
Singers are: "Ariana Grande", "Michael Jackson","Taylor Swift","XXXTentacion","Eminem", "Lady Gaga","Selena Gomez","Beyonce Knowles","Dua Lipa",'Jennifer Lopez.'
This dataset is not multi-label and does not consider a song from more than 2 singers.
Input texts are a title of a song and its lyrics concatenated with a new line character.





\subsection{c) spliting data}
\subsection{steps to build the dataset}

\section{Q2- Clustering}
\subsection{a) k=5 clusters}
\subsection{b) Examining clusters}
\subsection{c) Confusion matrix}
\subsection{d) Examining the confusion matrix}



\section{Q3- Comparing Classifiers}
\subsection{a) 5 baseline classifiers}
\subsection{b) My chosen classifier}

\section{Q4- Parameter Tuning}
\subsection{Classifier}
\subsection{Vectorizer}
\subsection{My own parameter}


\section{Context vectors using BERT}
\subsection{feature extraction and logistic regression}
\subsection{end-to-end trained classifier}
\subsection{Selecting hyperparameters and models}
\subsection{Examining models and parameters}


\section{Q6- Conclusions and Future Work}
















\end{document}